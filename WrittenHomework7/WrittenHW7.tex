\documentclass[11pt]{article}
\usepackage{verbatim, amsmath, amssymb, hyperref,enumerate,multicol}
\pagestyle{empty}
\voffset=-.5in
\textheight=8.75in
\textwidth=6in
\hoffset=-.35in

\newcommand{\ls}{\vspace{.1in}}
\newcommand{\ds}{\displaystyle}
\newcommand{\N}{\mathbb N}
\newcommand{\Z}{\mathbb Z}
\newcommand{\R}{\mathbb R}
\newcommand{\Q}{\mathbb Q}
\DeclareMathOperator{\RR}{{\it R}}
\newcommand{\abs}[1]{\lvert#1\rvert}


\usepackage{fancyhdr}
\pagestyle{fancy}
\lhead{Math Foundations}
\rhead{Written HW 7}
\cfoot{\thepage}
\renewcommand{\headrulewidth}{0.4pt}
\renewcommand{\footrulewidth}{0.4pt}

\begin{document}


\centerline{\bf Written Homework 7}

\centerline{Name(s): Elliot Marshall}

\begin{itemize}
\item[7.1.9.] Let $R$ be the relation on $\Z$ where for all $a,b\in \Z$, $a\RR b$ if and only if $\abs{a-b}\le 2$.

\begin{enumerate}[(a)]
\item Use set builder notation to describe the relation $\RR$ as a set of ordered pairs.
\par
\begin{center}
    $a\RR b = \{(a,b):a,b\subseteq \Z, -2 \leq a - b \leq 2\}$
\end{center}
\item Determine the domain and range of the relation $\RR$.
\par 
\begin{center} 
    Domain(R) = \{$u \in a | (u,y) \in \RR$ for at least one $y \in b$\} \\
    Range(R) = \{$v \in b | (x,v) \in \RR$ for at least one $x \in a$\}
\end{center}
\item Use the roster method to specify the set of all integers $x$ such that $x\RR 5$ and the set of all integers $x$ such that $5\RR x$.
\par 
\begin{center}
    $x\RR 5 = \{3,4,5,6,7\}$\\
    $5\RR x = \{3,4,5,6,7\}$
\end{center}
\item If possible, find integers $x$ and $y$ such that $x\RR 8$, $8\RR y$, but $x\not \!\!\RR y$.
\par
\begin{center}
    x = 6, y = 10
\end{center}
\item If $a\in \Z$, use the roster method to specify the set of all $x\in \Z$ such that $x\RR a$.
\par 
\begin{center}
    $x\RR a = \{x-2,x-1,x,x+1,x+2\}$
\end{center}
\end{enumerate}

\hrulefill

\item[7.2.15.] Define the relation $\approx$ on $\R\times\R$ as follows: For $(a,b),(c,d)\in\R\times\R$, $(a,b)\approx(c,d)$ if and only if $a^2+b^2=c^2+d^2$.
\begin{enumerate}[(a)]
\item Prove that $\approx$ is an equivalence relation on $\R\times\R$.
\par To prove that $\approx$ is an equivalence relation on $\R\times\R$,
\par it must be reflexive, symmetric, and transitive.
\begin{enumerate}
    \item[Reflexive:] let $(a,a)\in \R$. Then $a^2+a^2=a^2+a^2$ Then $(a,b)\approx(a,a)$
    \item[Symmetric:] let $(a,b)(c,d)\in \R\times\R$. Then $a^2+b^2=c^2+d^2$ and Thus $c^2+d^2=a^2+b^2$.
                        Therefore $(a,b)\approx(c,d)$ = $(c,d)\approx(a,b)$
    \item[Transitive:] let $(a,b),(c,d),(e,f)\in \R\times\R$. Then $a^2+b^2=c^2+d^2$ and $c^2+d^2=e^2+f^2$.
                        Therefore $a^2+b^2=e^2+f^2$ and $(a,b)\approx(c,d)$ and $(c,d)\approx(e,f)$ \\
                        Quorum Est Demonstrandum.
\end{enumerate}
\item List four different elements of the set 
\[
C=\{(x,y)\in\R\times\R\mid(x,y)\approx (4,3)\}.
\]
\begin{center}
    \begin{enumerate}
        \item[5:] (0,5) = $0^2 + 5^2 = 5^2 = 25$
        \item[4:] (4,3) = $4^2 + 3^2 = 16 + 9 = 25$
        \item[3:] (3,4) = $3^2 + 4^2 = 9 + 16 = 25$
        \item[$\sqrt{21}$:] (2,$\sqrt{21}$) = $2^2 + \sqrt{21}^2 = 4 + 21 = 25$
        \item[$\sqrt{24}$:] (1,$\sqrt{24}$) = $1^2 + \sqrt{24}^2 = 1 + 24 = 25$
    \end{enumerate}
\end{center}
\item Give a geometric description of the set $C$.
\par For each $x,y\in\Z$, $x^2 + y^2 =25$. This apperes to be a circle with radius $\sqrt{25}$ or $5$.

\end{enumerate}

\hrulefill

\item[7.3.7.] Define the relation $\sim$ on $\R$ as follows:
\begin{quote}
For $x,y\in\R$, $x\sim y$ if and only if $x-y\in\Q$.
\end{quote}
\begin{enumerate}[(a)]
\item Prove that $\sim$ is an equivalence relation on $\R$.
\par To prove that $\approx$ is an equivalence relation on $\R\times\R$,
\par it must be reflexive, symmetric, and transitive.
\begin{enumerate}
    \item[Reflexive:] let $x\in \R$. Then $x-x\in\Q$ Then $x\sim x$ which is true.
    \item[Symmetric:] let $x,y\in \R$. Then $x-y\in\Q$ and $y-x\in\Q$.
                        Therefore $x\sim y$ = $y\sim x$. This is also true.
    \item[Transitive:] let $x,y,z\in \R$. Then $x-y\in\Q$ and $y-z\in\Q$.
                        Therefore $x-z\in\Q$ and $x\sim y$ and $y\sim z$ therefore $x\sim z$.\\
                        Quorum Est Demonstrandum.    
\end{enumerate}
\item List four different real numbers that are in the equivalence class of $\sqrt 2$.
\begin{itemize}
    \item[1:] $\sqrt{2} + 1 \approx 2.4142 $
    \item[2:] $\sqrt{2} - 1 \approx 0.4142$
    \item[3:] $\sqrt{2} - 2 \approx -0.5858$
    \item[4:] $\sqrt{2} - 3 \approx -1.5858$
\end{itemize}
\item If $a\in\Q$, what is the equivalence class of $a$?
\par The equivalence class of $a$ for positive numbers is any number whose decimal is the same as $a$,
while for negetive numbers it is any negetive number whose decimal is the same as 1- $a$
\item Prove that $\left[\sqrt 2\right]=\left\{r+\sqrt 2\mid r\in\Q\right\}$.
\item If $a\in \Q$, prove that there is a bijection from $[a]$ to $\left[\sqrt 2\right]$.
\end{enumerate}

\end{itemize}



\end{document}