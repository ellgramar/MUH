\documentclass[11pt]{article}
\usepackage{verbatim, amsmath, amssymb, hyperref,enumerate,multicol}
\pagestyle{empty}
\voffset=-.5in
\textheight=8.75in
\textwidth=6in
\hoffset=-.35in

\newcommand{\ls}{\vspace{.1in}}
\newcommand{\ds}{\displaystyle}
\newcommand{\N}{\mathbb N}
\newcommand{\Z}{\mathbb Z}
\newcommand{\R}{\mathbb R}
\newcommand{\Q}{\mathbb Q}
\DeclareMathOperator{\RR}{{\it R}}
\newcommand{\abs}[1]{\lvert#1\rvert}


\usepackage{fancyhdr}
\pagestyle{fancy}
\lhead{Math Foundations}
\rhead{Written HW 7}
\cfoot{\thepage}
\renewcommand{\headrulewidth}{0.4pt}
\renewcommand{\footrulewidth}{0.4pt}

\begin{document}


\centerline{\bf Written Homework 7}

\centerline{Name(s):}

\begin{itemize}
\item[7.1.9.] Let $R$ be the relation on $\Z$ where for all $a,b\in \Z$, $a\RR b$ if and only if $\abs{a-b}\le 2$.

\begin{enumerate}[(a)]
\item Use set builder notation to describe the relation $\RR$ as a set of ordered pairs.
\item Determine the domain and range of the relation $\RR$.
\item Use the roster method to specify the set of all integers $x$ such that $x\RR 5$ and the set of all integers $x$ such that $5\RR x$.
\item If possible, find integers $x$ and $y$ such that $x\RR 8$, $8\RR y$, but $x\not \!\!\RR y$.
\item If $a\in \Z$, use the roster method to specify the set of all $x\in \Z$ such that $x\RR a$.

\end{enumerate}

\hrulefill

\item[7.2.15.] Define the relation $\approx$ on $\R\times\R$ as follows: For $(a,b),(c,d)\in\R\times\R$, $(a,b)\approx(c,d)$ if and only if $a^2+b^2=c^2+d^2$.
\begin{enumerate}[(a)]
\item Prove that $\approx$ is an equivalence relation on $\R\times\R$.
\item List four different elements of the set 
\[
C=\{(x,y)\in\R\times\R\mid(x,y)\approx (4,3)\}.
\]
\item Give a geometric description of the set $C$.

\end{enumerate}

\hrulefill

\item[7.3.7.] Define the relation $\sim$ on $\R$ as follows:
\begin{quote}
For $x,y\in\R$, $x\sim y$ if and only if $x-y\in\Q$.
\end{quote}
\begin{enumerate}[(a)]
\item Prove that $\sim$ is an equivalence relation on $\R$.
\item List four different real numbers that are in the equivalence class of $\sqrt 2$.
\item If $a\in\Q$, what is the equivalence class of $a$?
\item Prove that $\left[\sqrt 2\right]=\left\{r+\sqrt 2\mid r\in\Q\right\}$.
\item If $a\in \Q$, prove that there is a bijection from $[a]$ to $\left[\sqrt 2\right]$.
\end{enumerate}

\end{itemize}



\end{document}