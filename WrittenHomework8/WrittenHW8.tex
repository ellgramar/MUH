\documentclass[11pt]{article}
\usepackage{verbatim, amsmath, amssymb, hyperref,enumerate,multicol}
\pagestyle{empty}
\voffset=-.5in
\textheight=8.75in
\textwidth=6in
\hoffset=-.35in

\newcommand{\ls}{\vspace{.1in}}
\newcommand{\ds}{\displaystyle}
\newcommand{\N}{\mathbb N}
\newcommand{\nat}{\mathbb N}
\newcommand{\Z}{\mathbb Z}
\newcommand{\R}{\mathbb R}
\newcommand{\Q}{\mathbb Q}
\DeclareMathOperator{\RR}{{\it R}}
\newcommand{\abs}[1]{\lvert#1\rvert}
\DeclareMathOperator{\card}{card}


\usepackage{fancyhdr}
\pagestyle{fancy}
\lhead{Math Foundations}
\rhead{Written HW 8}
\cfoot{\thepage}
\renewcommand{\headrulewidth}{0.4pt}
\renewcommand{\footrulewidth}{0.4pt}

\begin{document}


\centerline{\bf Written Homework 8 (Chapter 9)}


\centerline{Name(s): Elliot Marshall}

\begin{itemize}
\item[9.1.5] Let $A$ and $B$ be sets. Prove that

\begin{enumerate}[(a)]
\item If $A$ is a finite set, then $A\cap B$ is a finite set.
\par $A\cap B$ contains all the elements of $B$ that are \underline{also} in $A$.
Thus, $A\cap B$ cannot contain any elements that are not in $A$; therefore, 
$|A\cap B|\leq |A|$. Which means that since A is a finite set, $A\cap B$ must 
also be a finite set
\item If $A\cup B$ is a finite set, then $A$ and $B$ are finite sets.
\par $A\cup B$ contains all the elements in either $A$ or $B$. Since $A\cup B $
is finite, $|A\cup B|$ must also be finite. Since $|A\cup B| > |A|$, 
$|A\cup B| > |B|$, and $|A\cup B|$ is finite, we can conclude that $A$ and $B$ are both finite.
\item If $A\cap B$ is an infinite set, then $A$ is an infinite set.
\par Since $A\cap B$ is an infinite set, and it contains the elements in both $A$ and $B$, 
we can conclude that $A$ (and $B$) is an infinite set as it contains an infinite number of elemens.
\item If $A$ is an infinite set or $B$ is an infinite set, then $A\cup B$ is an infinite set.
\par Since $A\cup B$ is made up of the elements in either $A$ or $B$, if $A$ is an infinite set, 
then it contributes an infinite number of elements to $A\cup B$. Likewise if $B$ is an infinite set, 
it too would contribute an infinite number of elements to $A\cup B$. In either case (or a combination
 of both), $A\cup B$ is an infinite set if at least one of its subsets is infinite.
\end{enumerate}

\hrulefill

\item[9.1.7] Prove the following proposition:
\begin{enumerate}
\item[(b)] If $A,B,C$, and $D$ are sets with $A\approx B$ and $C\approx D$, and if $A$ and $C$ are 
disjoint and $B$ and $D$ are disjoint, then $A\cup C\approx B\cup D$.
\par Since $A$ and $C$ are disjoint, $A\cap C = \emptyset$. Likewise, since $B$ and $D$ are disjoint,
$B\cap D = \emptyset$. Therefore since both $A\cap C$ and $B\cap D$ are equal to $\emptyset$, 
$A\cup C\approx B\cup D\approx \emptyset$
\end{enumerate}

\hrulefill

\item[9.2.2.] Prove that each of the following sets is countably infinite.

\begin{enumerate}[(a)]
\item The set $F^+$ of all natural numbers that are multiples of 5
\par Because the set $F^+$ is equivalent to multiples of 5, it can be rewritten as 
$F^+ = 5n$, such that $n\,\epsilon\,\N$. Therefore, since the natural numbers $(\N)$ are countably
 infinite, $F^+$ must also be countable infinite.
\item The set $F$ of all integers that are multiples of 5
\par Because the set $F^+$ is equivalent to multiples of 5, it can be rewritten as 
$F^+ = 5n$, such that $n\,\epsilon\,\Z$. Therefore since the integers $(\Z)$ are countably infinite, 
$F^+$ must also be countable infinite.
\item $\left.\left\{\dfrac{1}{2^k}\,\right\rvert\, k\in\N\right\}$
\par Because the set is based on the set of natural numbers and the set of natural numbers is countably 
infinite, the set is also countably infinite.
\item $\{n\in\Z\mid n\ge-10\}$
\par Similar to part C, since the set depends on the integers greater than or equal to -10 
(also known as $\{\N\} \cup \{-10,-9,...,-2,-1\}$) and the set of natural numbers plus the negetive 
numbers greater than -11 is countably infinite, the set listed above is also countably infinite.
\end{enumerate}

\hrulefill

\item[($\star$)]Optional Challenge: Replace one of the problems above with 9.2.9.

\item[9.2.9.] Define $f:\nat\times\nat\to\nat$ as follows: For each $(m,n)\in\nat\times\nat$, \[f(m,n)=2^{m-1}(2n-1).\]
\begin{enumerate}

\item Prove that $f$ is an injection. (See the textbook for hints.)
\item Prove that $f$ is a surjection. (See the textbook for hints.)
\item Prove that $\nat\times\nat\approx\nat$ and hence that $\card(\nat\times\nat)=\aleph_0$.

\end{enumerate}
\end{itemize}



\end{document}